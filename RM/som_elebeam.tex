
\subsection{Polarized Electron Beam}\label{sec:app_electronbeam}

The electron beam originated from a strained GaAsP 
photocathode illuminated by 
circularly polarized light~\cite{Sinclair2007}.
The sign of the laser polarization determined the electron
helicity; this was held constant for periods of 33 msec,
referred to as ``windows''.
By reversing the sign of the laser circular 
polarization, the direction of the spin at the target could be 
reversed rapidly \cite{Paschke:2007zz}.
Two windows of opposite helicity made a window pair, where
the helicity of the first window was chosen with a pseudorandom
number generator and the second window was the complement.
These window pairs were line locked to the 60 Hz line,
thus ensuring a good cancellation of the power-line noise. 
 
A half-wave ($\lambda$/2) plate was periodically inserted into the 
laser optical path which passively reversed the
sign of the electron beam polarization. 
Roughly equal statistics were thus 
accumulated with opposite signs for the measured asymmetry, 
which suppressed many systematic effects.  
The direction of the polarization could also be
controlled by a Wien filter and solenoidal lenses
near the injector \cite{GramesWien2011}.  The accelerated beam was 
directed into Hall A, where its intensity, energy and trajectory on 
target were inferred from the response of several monitoring devices.

The beam monitors and trigger signals, which derived from
the detectors in the spectrometers, 
were integrated over the helicity window
and digitized.  The beam monitors were integrated by
custom 18-bit ADCs ~\cite{ref:prex}, see section ~\ref{sec:daq}.
The beam monitors, target, and detectors were designed so that
the fluctuations in the fractional difference in 
the signal response between
a pair of successive windows were
dominated by scattered electron counting statistics.
To keep spurious beam-induced asymmetries under
control at well below the ppm level, 
careful attention was given to the design and configuration of the laser 
optics leading to the photocathode \cite{Paschke:2007zz}.

The integrated response of each detector PMT and beam monitor
was digitized and recorded for each 33 msec window.
The raw spin-direction asymmetry $A_{raw}$
in each spectrometer arm was computed from the the detector response
normalized to the beam intensity 
for each window pair. 

