\subsection{Systematic Fluctuations and Beam Corrections}
\label{sec:beamcorr}

In this section we consider possible corrections from
the helicity-correlations in the beam.

If one considers the cumulative corrected asymmetry
$\acorr_d$ over many window pairs, one can write
\begin{eqnarray}
\acorr_d & \equiv & \langle (\acorr_d)_i\rangle = 
\nonumber \\
&  & 
\left\langle\left(\frac{\Delta D}{2D}\right)_i\right\rangle -
\left\langle\left(\frac{\Delta I}{2I}\right)_i\right\rangle
- \sum_j {\beta_j\left\langle{(\Delta M_j)_i}\right\rangle} 
\nonumber \\
&=& \asy_D -  \asy_I - \sum_j \asy_{Mj}.
\end{eqnarray}

For most of the running conditions during data collection,
$\acorr_d\simeq \asy_D\simeq 10$~ppm, which meant that all
corrections were negligible. The cumulative average for $\asy_I$ was
maintained below 0.1 ppm. For $\asy_{Mj}$, the cumulative averages
were found to be below 0.1 ppm during the run.
This resulted from the fact that the accelerator
damped out position fluctuations produced at the source by a large
factor ~\ref{happex}.

Adjustments of the circular polarization of the
laser beam was required to reduce the differences to about 0.1 $\mu$m. 
This resulted in observed position differences on target
ranging from 10 nm to 100 nm, which in turn resulted in $\asy_{Mj}$
in the range from 0.1 to 1 ppm.
During the run, the control of the asymmetry corrections 
within the required limits was accomplished with feedback on 
the laser and electron beam properties in order to maintain 
small helicity correlations; these methods are 
discussed in ref ~\ref{happex}.
