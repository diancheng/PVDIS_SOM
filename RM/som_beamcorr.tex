\subsection{Systematic Fluctuations and Beam Corrections}
\label{sec:beamcorr}

Assuming that $\sigma(\asy_d)$ has negligible contributions from
window-to-window beam fluctuations and instrumentation noise,
there is still the possibility that there are 
helicity-correlated systematic effects at the sub-ppm
level. If one considers the cumulative corrected asymmetry
$\acorr_d$ over many window pairs, one can write
\begin{eqnarray}
\acorr_d & \equiv & \langle (\acorr_d)_i\rangle = 
\nonumber \\
&  & 
\left\langle\left(\frac{\Delta D}{2D}\right)_i\right\rangle -
\left\langle\left(\frac{\Delta I}{2I}\right)_i\right\rangle
- \sum_j {\beta_j\left\langle{(\Delta M_j)_i}\right\rangle} 
\nonumber \\
&=& \asy_D -  \asy_I - \sum_j \asy_{Mj}.
\end{eqnarray}

For most of the running conditions during data collection,
$\acorr_d\simeq \asy_D\simeq 10$~ppm, which meant that all
corrections were negligible. The cumulative average for $\asy_I$ was
maintained below 0.1 ppm. For $\asy_{Mj}$, the cumulative averages
were found to be below 0.1 ppm during the run with the ``bulk'' GaAs
photocathode. This resulted from the fact that the accelerator
damped out position fluctuations produced at the source by a large
factor (section \ref{sec:adiabatic_damp}). 
The averaged position differences 
on target were kept below 10 nm.

However, during data collection with ``strained'' GaAs, position
differences as large as several $\mu$m were observed in the
electron beam at a point in the accelerator where the beam energy is
5 MeV. Continuous adjustment of the circular polarization of the
laser beam was required to reduce the differences to about 0.5
$\mu$m. This resulted in observed position differences on target
ranging from 10 nm to 100 nm, which in turn resulted in $\asy_{Mj}$
in the range from 0.1 to 1 ppm.

The control of the asymmetry corrections within the aforementioned
constraints was one of the central challenges during data
collection. A variety of feedback techniques on the laser and
electron beam properties were employed in order to accomplish
this; these methods are discussed in Sec.~\ref{sec:laser}.
