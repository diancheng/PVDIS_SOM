
\section{Apparatus}\label{sec:apparatus}

\par The experimental techniques for measuring small 
asymmetries of order 1 ppm have been successfully deployed in
parity-violating electron-scattering experiments at several
facilities \cite{SLAC,bates,mainz1,happex,g0,prex,qweak,mainz2}.
The recent experiments at Jefferson Lab, such as HAPPEX ~\cite{happex}
and PREX ~\cite{prex} have maintain systematic errors associated with helicity
reversal at the $10^{-8}$ level.
The asymmetries sought for in this experiment are of order 100 ppm with
required accuracies of about 1 ppm, which is two orders-of-magnitude 
above the systematic error established in the recent experiments.

The floor plan for Hall A is shown in figure ~\ref{fig:floorplan}.
We used an 85 $\mu$A polarized electron beam and a 25 cm liquid
deuterium target.  The scattered electrons are detected by the two
High Resolution Spectrometers (HRS) in Hall A at Jefferson Lab.
A Luminosity Monitor is located downstream on the beamline to monitor 
the target boiling effects and possible false asymmetries at the $10^{-7}$ 
level.  

A significant challenge of the measurement 
is to separate electrons from the charged pion background 
that arise from electro- or photo-production. 
While the standard HRS detector package and data 
acquisition (DAQ) system routinely provide 
such a high particle identification (PID) performance, 
they are based on full recording 
of the detector signals and are limited to event rates up to 4 kHz.
This is not sufficient for the few-hundred kHz rates 
for the experiment. 
Thus we have built new DAQ designed to count event 
rates up to 1~MHz with hardware-based 
particle identification (see ref ~\cite{pvdis_nim} and section \ref{sec:daq}).

The apparatus will be described in detail in this section.
These include the polarized electron beam
(section ~\ref{sec:elebeam}), the beam monitors (\ref{sec:beam_moni}), 
the spectrometers and detectors(\ref{sec:hrs}), 
the data acquisition system (\ref{sec:daq}, 
the target (\ref{sec:target}),
and the beam polarimeters (\ref{sec:beam_pol}).


