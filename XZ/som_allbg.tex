
\subsection{Background Analysis}\label{sec:ana_allbg}

{\bf numbers awaiting inputs:}

\begin{itemize}
\item pion and electron asymmetry results from wide triggers;

\item analytic estimation of OR, for left kine 2 - Kai (this goes to NIM);

\item e in pion contamination analysis, need to check results, and need all detector efficiency numbers - Kai (this goes to NIM);

The above results will affect the pion asymmetry (corrected for electron contamination), which further affect
1) the uncertainty in the main results due to pion contamination, and 2) evaluation of positron results; 

\item Run-average deadtime correction for unblinding - Diancheng
\end{itemize}

\subsubsection{Charged Pion Background}

Charged pions are produced from decays of nucleon resonances created by 
electron scattering off nucleon or nuclear targets. For the pions to have the same
momentum as DIS electrons, the parent nucleon resonance production must occur at 
a lower $Q^2$ than DIS events, causing a smaller parity violation asymmetry than DIS electrons.
This has been confirmed by the asymmetry of the pion triggers measured during the
experiment. Furthermore, the high particle identification performance of the
customized DAQ limited the pion contamination in the electron trigger to the
$f=2\times 10^{-4}$ level or below~\cite{pvdis_nim}. Due to the small contamination, 
effect of the pions was considered a dilution and no correction to the measurement
electron was made. The total systematic uncertainty on the electron 
DIS asymmetry due to pion contamination and pion asymmetry is:
\begin{equation}
 \Delta A = \sqrt{\left(\Delta f\right)^2+\left(f\frac{\vert{A_\pi}\vert+\Delta A_\pi}{A_e}\right)^2}~\label{eq:pionbg}
\end{equation}
where $f$ is the event fraction of the electron trigger that is from pions, $A_\pi$ is the 
measured pion asymmetry with $\Delta A_\pi$ its uncertainty, and $A_e$ is the measured electron
asymmetry. The term $\vert A_\pi\vert+\Delta A_\pi$ corresponds to how much the pion asymmetry differs
from zero. Results for $f$ and its error bars are presented in Ref.~\cite{pvdis_nim}. Extraction of
the pion asymmetry from the pion trigger asymmetry is described below. 
The measured electron asymmetries before any corrections is
made are $78.4$ and $140.5$~ppm for $Q^2=1.1$ and $1.9$~GeV$^2$, respectively.

\bigskip
\noindent
{\bf pion asymmetry measurement}

PID performance of both electron and pion triggers of the DAQ was reported in Ref.~\ref{pvdis_nim}. 
To properly extract pion asymmetries from the trigger, one must properly account for the effect
of electron contamination. Because electron contamination in the pion triggers, $f_{e/\pi}$, was
relatively high and the electron asymmetries are larger than those of pions, corrections were applied
to the (raw) asymmetries extracted from the pion triggers using
\begin{eqnarray}
  A_\pi^{corrected} = \frac{A_\pi^{raw}-f_{e/\pi}A_e}{1-f_{e/\pi}}~,
\end{eqnarray}
where $A_e$ is the electron asymmetry provided from the electron triggers. 

\bigskip
\noindent
{\bf electron asymmetry uncertainty due to pion contamination}

Results for the pion contamination in electron triggers $f_{\pi/e}$ and the electron contamination
in pion triggers $f_{e/\pi}$ and their total uncertainties are shown in Table~\ref{tab:pionbg}. 
These results were reported in Ref.~\ref{pvdis_nim}.
Also shown are the the raw and the corrected pion asymmetries, and the 
total uncertainty on the electron asymmetry due to pion contamination as calculated from Eq.~\ref{eq:pionbg}.

\begin{table}
 \caption{Pion asymmetry results and total uncertainty on electron asymmetry due to pion background.}\label{tab:pionbg}
 \begin{tabular}{c|c|c|c}\hline
  HRS                 & Left & Left & Right\\
  $Q^2$ (GeV/$c$)$^2$ & 1.1 & 1.9 & 1.9 \\\hline
  \multicolumn{4}{c}{narrow path}\\\hline
  raw $A_\pi$ (ppm) & $-45.79\pm 7.98$(stat.) & $-14.00\pm 14.89$(stat.) & $-9.51\pm 4.22$(stat.) \\
  $f_{e/\pi}$ & $0.564\pm 0.159$(syst.) & $0.0403\pm 0.0033$(syst.) & $0.0112\pm 0.0013$(syst.) \\
% not the latest   corrected $A_\pi$ (ppm) & $-30.85\pm 12.84$(tot.) & $-8.91\pm 16.31$(tot.) & $-8.04\pm 4.27$(tot.)\\
  corrected $A_\pi$ (ppm) & $$(tot.) & $$(tot.) & $$(tot.)\\
  $A_e$ measured (ppm) & 78.4 & 140.5 & 140.5 \\
  $f_{\pi/e}$ & $(1.618\pm 0.291)\times 10^{-4}$ & $(2.196\pm 0.345)\times 10^{-4}$ & $(1.987\pm 0.333)\times 10^{-4}$ \\
% not the latest  ${\Delta A}\over{A}$ & $0.0000947$ & $0.0000524$ & $0.0000376$ \\\hline
  ${\Delta A}\over{A}$ & $$ & $$ & $$ \\\hline
  \multicolumn{4}{c}{wide path}\\\hline
  raw $A_\pi$ (ppm) & $$(stat.) & $$(stat.) & $$(stat.) \\
  $f_{e/\pi}$ & $$(syst.) & $$(syst.) & $$(syst.) \\
  corrected $A_\pi$ (ppm) & $$(tot.) & $$(tot.) & $$(tot.)\\
  $A_e$ measured (ppm) & 78.4(?) & 140.5(?) & 140.5(?) \\
  $f_{\pi/e}$ & $(1.009\pm 0.240)\times 10^{-4}$ & $(1.833\pm 0.308)\times 10^{-4}$ & $(1.588\pm 0.314)\times 10^{-4}$ \\
  ${\Delta A}\over{A}$ &           &           &           \\\hline
 \end{tabular}
\end{table}


\subsubsection{Pair Production Background}

Pair production background comes from nucleon resonance productions where the resonance
decays into $\pi^0$'s, then through pion decay $\pi^0\to e^+e^-$. 
The pair production from Bremsstrahlung photons is highly
forward-peaked and is not significant for the kinematics proposed here.
One therefore expect
the pair production background to have an asymmetry that is comparable to the charged
pion asymmetry reported above. This background was studied during the experiment by 
reversing the spectrometer polarity, allowing detection of the positron alone in the $\pi^0$ decay. 
The main focus of such positive polarity runs (or ``positron runs'') is to precisely
determine the fractional contribution from pair production to the main electron trigger,
$f_{e^+/e^-}$. Due to the relative low rate of positron events, this ratio can be extracted
from the regular DAQ of which the PID performance and rate determination were well understood. 
Asymmetry of positrons was recorded, although due to the very low rate of positrons the 
uncertainty of such asymmetry measurement is large. Results for the asymmetry
extracted from positive polarity runs (using the electron triggers of the DAQ, which are now
effectively positron triggers) are shown in Table~\ref{tab:Apositron}.
Note that there is a large $\pi^+$ contamination in the positron trigger but there
was not enough data to correct this $\pi^+$ background.

\begin{table}
 \caption{Positron asymmetry results.}\label{tab:Apositron}
 \begin{center}
\begin{tabular}{c|c|c}\hline
  HRS                 &{Left} & {Right}\\\hline
  $Q^2$ (GeV/$c$)$^2$ & {1.1} & {1.9}\\\hline
  raw $A_{e^+}$ (ppm), narrow & $723.2\pm 1154.7$(stat.) & $1216.0\pm 1304.5$(stat.) \\
  raw $A_{e^+}$ (ppm), wide   & $742.4\pm 1151.5$(stat.) & $1199.0\pm 1304.5$(stat.)\\\hline % elog entry #107, latest
\end{tabular}
 \end{center}
\end{table}

Because the statistical uncertainties of the positron asymmetry results are large, we relied on the fact
that $\pi^0$ must have similar asymmetry as $\pi^-$. We assume the $\pi^0$ asymmetry to be no larger than twice
the value of $\pi^-$ asymmetry and estimate the uncertainty in the electron asymmetry due to positron background 
as:

\begin{equation}
 \Delta A = \sqrt{\left(\Delta f_{e^+/e^-}\right)^2+\left(f_{e^+/e^-}\frac{\Delta{A_{e^+}}}{A_e}\right)^2}~,~\label{eq:posbg}
\end{equation}
where $\Delta A_{e^+}$ describes how much $A_{e^+}$ differs from zero and the value 
$2(\vert A_\pi\vert +\Delta A_\pi)$ was used. Results for $f_{e^+/e^-}$ and their
statistical errors are shown in Table~\ref{tab:positronbg}, and a $10\%$ systematic uncertainty is used
for $\Delta f_{e^+/e^-}$ to account for possible error in positron identification from the high $\pi^+$ background 
in the rate evaluation.
Results for the electron asymmetry uncertainty due to pair production background 
are also shown in Table~\ref{tab:positronbg}.
\begin{table}
 \caption{Results for positron contamination $f_{e^+/e^-}$ and total uncertainty on electron asymmetry 
due to pair production background. The errors shown for $f_{e^+/e^-}$ are statistical only, and a 10\%
systematic uncertainty was used in the evaluation of $\Delta A\over A$.}\label{tab:positronbg}
\begin{tabular}{c|c|c|c}\hline
  HRS                 & Left & Left & Right\\\hline
  $Q^2$ (GeV/$c$)$^2$ & 1.1 & 1.9 & 1.9\\\hline
  $f_{e^+/e^-}$         & $(2.504\pm 0.007)\times 10^{-4}$ & $(5.154\pm 0.001)\times 10^{-3}$
                      & $(4.804\pm 0.001)\times 10^{-3}$\\\hline
  ${\Delta A}\over{A}$ & $2.504\times 10^{-5}\circ{+}???$ & $5.154\times 10^{-4}\circ{+}???$
                      & $4.804\times 10^{-4}\circ{+}???$ \\\hline
\end{tabular}
\end{table}


\subsubsection{Target EndCap Correction}

Electrons scattered off the target aluminum endcaps cannot be separated from those scattered off
the liquid deutrium. Fortunately events from target endcaps also belong to deep inelastic scattering
and one expect the DIS formula for asymmetries (refer to equation in ``formalism'') to work for aluminum
as well. Since Al is not an isoscalar nucleus, the Al PVDIS asymmetry differs from the deuterium and 
a correction must be made. 
Possible deviations for $A_{Al}$ to differ from the DIS formula is the nuclear effect similar to the EMC effect
of the unpolarized, parity-conserving structure functions $F_{1,2}$, but one does not expect such effect
to cause more then 20\% difference to $A_{Al}$ and this assumption will be used in the uncertainty estimation.

The fractional event rate from the aluminum endcaps, $f_{Al/D}$, is assumed to be equal to the ratio of the endcap
to liquid deuterium thickness, $\eta_{Al/D}$. 
This is based on the assumption that the DIS cross section from Al is the same as that from D$_2$. 
Using current data on the EMC effect, the difference in 
DIS cross sections between Al and LD$_2$ indeed should be very small since the EMC ratio crosses unity
between $x=0.2$ and $0.3$, exactly where data were taken during this experiment. 
The target used for this experiment had entrance and exit endcaps measured to be $0.126\pm 0.011\pm 0.003$~mm
and $0.100\pm 0.008\pm 0.003$~mm respectively, with the first error bar from the standard deviation of 
multiple measurements at difference positions on the endcap, and the second error from calibration of
the instrument. The ratio $\eta_{Al/D}$ is evaluated to be 
$\eta_{Al/D}=(0.126+0.100)$~mm$\times 2.7$~g/cm$^3)/(20$~cm$\times 0.167$~g/cm$^3)=1.827\%$ with a total
error of 0.115\%.

The correction to the electron PVDIS asymmetry is applied as
\begin{eqnarray}
  A^{corrected} &=& A^{measured}(1+\delta_{Al}),~~\mathrm{with}~\delta_{Al} = -(\eta_{Al/D})\frac{A_{Al}-A_{D}}{A_D}.
\end{eqnarray}
The DIS formalism was used to evaluate the Al PVDIS asymmetry as:
\begin{eqnarray}
  A_{Al} &=& \frac{13A_p\sigma_p+14A_n\sigma_n}{13\sigma_p+14\sigma_n}~,
\end{eqnarray}
where $\sigma_{p(n)}$ is the DIS cross section off the proton (neutron) as calculated from structure function
fits and $A_{p(n)}$ is the PVDIS asymmetry off the proton (neutron):
\begin{eqnarray}
  A_p &=& \left(-\frac{3 G_FQ^2}{2\sqrt{2}\pi \alpha}\right)
    \frac{Y_1\left[2C_{1u}(u^++c^+)-C_{1d}(d^+ +s^+)\right]
         +Y_3\left[2C_{2u}(u^-)-C_{2d}(d^-)\right]}
         {4(u^++c^+)+(d^++s^+)}\\
  A_n &=& \left(-\frac{3 G_FQ^2}{2\sqrt{2}\pi \alpha}\right)
    \frac{Y_1\left[2C_{1u}(d^++c^+)-C_{1d}(u^++s^+)\right]
         +Y_3\left[2C_{2u}(u^-)-C_{2d}(d^-)\right]}
      {4(d^++c^+)+(u^++s^+)}
\end{eqnarray}
with $u^\pm\equiv u\pm \bar u$, $d^\pm\equiv d\pm\bar d$, $s^+\equiv s+\bar s$ and $c^+\equiv c+\bar c$.

The total uncertainty due to target endcaps, assuming an up to 10\% difference in the Al vs. D$_2$ PVDIS
asymmetry, is approximately
\begin{eqnarray}
  \frac{\Delta A}{A} &=& \sqrt{\left(\Delta\eta_{Al/D}\right)^2+\left((10\%)\eta_{Al/D}\right)^2}
\end{eqnarray}
where the second term dominates.
Results for the endcap correction $\delta_{Al}$ and the uncertainty on the corrected electron asymmetry are
given in Table~\ref{tab:albg}.
\begin{table}
 \caption{Correction applied to the measured asymmetry to account for the target
aluminum endcaps.}\label{tab:albg}
 \begin{center}
\begin{tabular}{c|c|c}\hline
  $Q^2$ (GeV/$c$)$^2$ & 1.1 & 1.9 \\\hline
  $(A_{Al}-A_D)/A_D$  & $0.005670$ & $0.007268$ \\
  $\delta_{Al}$       & $-1.036\times 10^{-4}$ & $-1.328\times 10^{-4}$ \\
  ${\Delta A}\over{A}$ & $0.001827$ & $0.001827$\\\hline
\end{tabular}
 \end{center}
\end{table}


Events were also taken on a thick, ``dummy'' target consists of two thick aluminum endcaps the thickness
approximately 10 times that of the liquid deuterium cell. The thickness was chosen such that the total radiation
length of the dummy target matches that of the liquid D$_2$ target. However, due to limited beam time, 
the asymmetry uncertainty collected from the aluminum dummy target was not precise enough to reduce
the systematic uncertainty due to target endcaps.

\subsubsection{Transverse Asymmetry Background}

Transverse asymmetry background describes the effect of the electron beam spin polarized in the direction
perpendicular (normal) to the scattering plane defined by the momentum vectors of the incident and the scattered 
electrons $\vec k_e$ and $\vec k_e^\prime$. 
This beam normal asymmetry is parity violating but differs from the PVDIS asymmetry caused
by the longitudinal polarization of the beam, thus must be treated as a background of the measurement.
Calculations at the pure partonic level show that this asymmetry is at the 0.5~ppm level at the kinematics
of this experiment, but mechanism 
beyond the parton level can enhance the asymmetry by 1-2 orders of magnitude~(cite Andrei Afanasev priv communication
latest email 10/29/2012). Contribution from the beam normal asymmetry $A_n$ to the measured asymmetry 
can be expressed as
\begin{eqnarray}
  \delta A = (A_n) \vec S\cdot \hat k_n~&\mathrm{with}&\vec k_n\equiv \hat k_e\times \hat k_e^\prime~~\mathrm{and}~
\hat k_n=\vec k_n/\vert\vec k_n\vert~,
\end{eqnarray}
were $\vec S$ is the beam polarization vector.
Denoting $\theta_0$ the central scattering angle of the spectrometer and $\theta_{tr}$ the average
out-of-scattering-plane angle of the spectrometer acceptance as defined in Fig.~\ref{fig:ATkine}, 
one has $\hat k_e=(0,0,1)$ and 
$\hat k_e^\prime = (\sin\theta_0\cos\theta_{tr}, \sin\theta_0\sin\theta_{tr},\cos\theta_0)$~, 
giving $\vec k_n=(-\sin\theta_0\sin\theta_{tr},\sin\theta_0\cos\theta_{tr},0)$
and $\hat k_n = (-\sin\theta_{tr},\cos\theta_{tr},0)$, thus
\begin{eqnarray}
 \delta A &=& {A_n}\left[-S_H\sin\theta_{tr}+S_V\cos\theta_{tr}\right]
\end{eqnarray}
where $S_V$, $S_H$ and $S_L$ are respectively the electron's spin polarization components in the vertical,
horizontal, and longitudinal directions. The value of $\theta_{tr}$ was determined
from the simulation and was found to be less than 0.01~rad. Since the beam spin during production
runs was controlled to $S_H<20\% P_b$ and $S_V<2\% P_b$ where $P_b$ is the beam polarization,  
the $S_V$ term dominates the effect on the measured asymmetry.

\begin{figure}
 \caption{Kinematics of the transverse asymmetry background. The incident and the scattered electrons'
momenta are $\vec k_e$ and $\vec k_e^\prime$, and $\vec S_{V,H,L}$ denote respectively 
the incident electron's spin polarization
components in the vertical, horizontal, and longitudinal directions. 
The central scattering angle of the spectrometer is $\theta_0$ and the scattered electron's momentum
has an out-of-plane angle denoted by $\theta_{tr}$.}\label{fig:ATkine}
 \begin{center} 
  \includegraphics[scale=1.0]{XZ/AT_kine.eps}
 \end{center}
\end{figure}


During the experiment, the size of the beam transverse asymmetry was measured during dedicated ``transverse
runs'' where the beam was fully polarized in the vertical direction, $S_H=S_L\approx 0$ and $S_V=P_b^T$ where
$P_b^T$ is the beam polarization during the transverse asymmetry measurement. 
The measurement thus provides the value $A_n^{m}=A_n P_b^{T}$. Since the maximum beam polarization 
is the same for production and transverse asymmetry running, and the horizontal component of the beam
spin $S_H$ is no more than 20\% during the production runs, the longitudinal beam polarization during production
running, $S_L$, cannot differ from $P_b^T$ by more than $1-\sqrt{1-(20\%)^2}=1.01\%$. The total uncertainty
in the PVDIS electron asymmetry can be estimated using
\begin{eqnarray}
 \frac{\Delta A}{A} &=& \frac{(\vert A_n^m\vert +\Delta A_n^m)}{A_e^m}\sqrt{\left[ S_V^2+(1.01\%)^2\right]} 
= 0.0224\frac{\delta A_n^m}{A_e^m}
\end{eqnarray}
where $\delta A_n^m$ describes how much $A_n$ could differ from zero and is taken to be
$\Delta A_n^m$ if the measured asymmetry is consistent with zero and $(\vert A_n^m\vert +\Delta A_n^m)$ otherwise;
$A_e^m$ is the 
measured PVDIS electron asymmetry.

Results for
the beam transverse asymmetry measurements are shown in Table~\ref{tab:ATbg} along with the 
total uncertainty on the electron PVDIS asymmetry results duto beam transverse polarizations.

\begin{table}
 \caption{Results from the dedicated beam transverse asymmetry measurements and estimation
of the total uncertainty on the PVDIS electron asymmetry due to beam transverse polarization.}\label{tab:ATbg}
 \begin{center}
\begin{tabular}{c|c|c}\hline
  HRS                  & Left & Right \\
  $Q^2$ (GeV/$c$)$^2$  & $1.1$ & $1.9$ \\\hline
  $A_n^m$ (ppm, narrow)& $-24.15\pm 15.05$ & $23.49\pm 44.91$ \\
  $A_n^m$ (ppm, wide)  & $-24.66\pm 15.01$    & $24.60\pm 44.90$\\
  $A_e^m$ (ppm)       & $78.4$ & $-140.5$ \\
  ${\Delta A}\over{A}$ & $0.014195$ & $0.00717$\\\hline
\end{tabular}
 \end{center}
\end{table}


Total error previously quoted: 0.34\%, 0.56\% (a factor of $\sin\theta_0$ was used which suppressed the effect).

\subsubsection{Target Purity, Density Fluctuation and Other False Asymmetries}
\label{sec:syst_purity}

{\bf the following is copied from the proposal and must be revised.}\\

% from the proposal
The liquid deuterium used contains~\cite{target:purity} $1889$~ppm 
HD, $<100$~ppm H$_2$, $4.4$~ppm N$_2$, $0.7$~ppm O$_2$, $1.5$~ppm CO 
(carbon monoxide), $<1$~ppm methane and $0.9$~ppm CO$_2$ (carbon dioxide).  
Compared to the statistical accuracy of the measurement ($\approx 1.1\%$ 
in $A_d$), the only non-negligible contamination to the measured asymmetry 
is from the proton in HD. Since the asymmetry of the proton is given 
by~\cite{exp:e149_new}
\begin{eqnarray}
 A_p &=& \big(\frac{3G_F Q^2}{\pi\alpha 2\sqrt{2}}\big)
	\frac{2C_{1u}u(x)-C_{1d}[d(x)+s(x)]+Y[2C_{2u}u_v(x)-C_{2d}d_v(x)]}
	     {4u(x)+d(x)+s(x)} \label{equ:Asymp}
\end{eqnarray}
which is within $\pm 10\%$ of the asymmetry of the deuteron, the proton 
in HD contributes $\delta{A_d}/A_d<0.02\%$ uncertainty to the measured asymmetry.
%
%\smallskip
The Luminosity Monitor can make sure that the target density fluctuation is 
less than $0.1$~ppm.  This is $<0.1\%$ uncertainty in the measured asymmetry.


% from the proposal
\subsection{Rescattering Background}\label{sec:syst_background}

{\bf the following is copied from the proposal and must be revised.}\\

The rescattering of high-energy electrons or pions from the walls of 
the spectrometer creates a potential source of background for the proposed 
measurement.   This ``rescattering'' background, which is typically 
rejected using a combination of tracking and particle 
identification in low-rate experiments without difficulty, must be treated
carefully in this high-rate measurement due to the limited information 
available in each event. 

The magnitude of this effect will be combination of the
probability for products of this scattering in the spectrometer to
reach the detectors and the effectiveness of the 
detector/DAQ package to distinguish those tracks from tracks 
originating in the target.  A detailed analysis of this possible 
problem will require a careful simulation of the spectrometer and 
detector geometry.  Measurements will be taken
with a low beam current (to allow the use of the tracking
chambers and the standard DAQ) to study this small background and
verify the accuracy of the simulation.

Rescattering contribution has been studied by the previous HAPPEX II experiments
in Hall A (HAPPEX-H: E99-115 and HAPPEX-He: E00-114)~\cite{HAPPEX-2005}. 
These experiments used an analog-integrating detector, and therefore 
had no method for excluding rescattered background particles. The method used 
includes a series of dedicated elastic scattering measurements with a hydrogen target, 
with the spectrometer tuned to place the hydrogen elastic peak at various points inside 
the spectrometer.  The detected rate was used to estimate the 
``rescattering probability'': the probability that an electron, 
interacting at a given point in the spectrometer, 
produces a count in the production DAQ. 
In those measurements, the rescattering probability was found to be around 1\% 
for momenta near to the central momenta (within a few percent of 
$\delta p/p$). This probability rapidly dropped to $10^{-5}$ for 
interactions with the spectrometer wall took place before the last 
spectrometer quadrupole element (Q3). For HAPPEX-He, the rate of 
quasi-elastic scattering from the Helium target which was steered 
into the spectrometer walls was several times the elastic
signal rate, leading to a rescattering in the focal plane on order 0.2\% of the
detected elastic rate. It is reasonable to expect that the detected 
rescattering signal in the proposed measurement will also form a dilution 
at the few $10^{-3}$ level. Factors that would argue for a larger contribution, 
such as the continuous DIS momentum distribution and the relatively open 
spectrometer geometry, will be counteracted by the ability to exclude 
background through position, energy, or PID information from the fast 
counting DAQ. 

One also have background from pion rescattering. 
However, pions can be rejected by PID detectors reliably and will only have
negligible contribution to the primary measurement. 

Overall, we expect that the total rescattering rate to be at most at a few 
$\times 10^{-3}$ level. And among these rescattered events, resonance 
electrons and pions will only consist a small fraction. The rescattered
DIS electrons may be the majority of these rescattering events but they have 
very similar kinematics and $Q^2$ to the primary measurement thus will only 
introduce a very small 
dilution. Therefore we expect the total uncertainty due
to the rescattered background to be in the $10^{-4}$ range.
