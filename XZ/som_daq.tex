\subsubsection{Data Acquisition System}
\label{sec:daq}

The signals from our trigger that define electrons and 
pions are sent to scalers (Struck Model 3801)
where they are integrated over the helicity window.  
The scalers are part of the HAPPEX DAQ ~\ref{ref:happex}
which is a multiple-VME-crate DAQ system running under the CODA system
developed at Jefferson Lab ~\cite{coda}.  
Signals from the various beam monitors 
are integrated and digitized by custom-built 
VME integrating 18-bit ADCs.  The system is
triggered at the 30 Hz rate of the helicity reversal,
synchronized to the end of each helicity window
with the first 0.5 msec of the pulse 
blanked off to remove instabilities due to the switching of
HV on the Pockels cell which controls the beam polarization.
In addition to the scalers and ADCs, the DAQ reads
input/output registers which record various information 
such as the helicity.

The scaler DAQ which counts triggers is designed
to count event rates up to 1~MHz with hardware-based PID
and with minimal deadtime of order 1\%.  
The analysis
of the deadtime is given in reference ~\ref{pvdis_nim}.
The detectors (section ~\ref{sec_detector_system})
provided the electron and pion triggers.  A schematic 
and full description of the
trigger is shown in ref ~\ref{pvdis_nim}.


